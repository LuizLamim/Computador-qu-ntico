\documentclass[12pt, a4paper]{letter} % Set the font size (10pt, 11pt and 12pt) and paper size 
\usepackage{graphicx}
\usepackage{geometry}
\usepackage{eso-pic}
\usepackage[export]{adjustbox}
\usepackage{tikz} 

\usepackage[utf8]{inputenc}
\usepackage{longtable} % Required for tables that span multiple pages


\usepackage{xcolor}
\usepackage{transparent}
\definecolor{umdred}{RGB}{77, 95, 227}

\geometry{
	top=1cm,
	bottom=2cm,
	left=2cm,
	right=2cm,
}

\newcommand\BackgroundPicture{
	\put(0,0){
		\parbox[b][\paperheight]{\paperwidth}{
			\vfill
			\centering
			\transparent{0.15}\includegraphics[width=0.5\paperwidth,height=0.5\paperheight,keepaspectratio]{IRB.png}
			\vfill
}}}
\AddToShipoutPicture{\BackgroundPicture}

\begin{document}


\begin{minipage}{0.6\textwidth}
	\includegraphics[width=3in]{IRB.png}
\end{minipage}
\hfill
\begin{minipage}{0.4\textwidth}\raggedright
	\small{\textbf{\color{umdred}Site:}\hphantom{A}https://irb.academy/\\
		\textbf{\color{umdred}Endereço:}\hphantom{A} Ed. Jade Home Office\\
		\hphantom{AA}SGCV/S, Lote 15, Bloco C, St.\\
		\hphantom{AA}Park Sul\\
		\hphantom{AA}Sala 224, Brasília\\
		\hphantom{AA}DF, 71215-100\\
		\textbf{\color{umdred}WhatsApp: }(61) 8187-4673\\
		\textbf{\color{umdred}Email: }suporte@irb.academy }
\end{minipage}

\today

\maketitle
Pauta – Reunião sobre Possível Migração do Tutor MSL para Moodle

Gostaria de abrir um ponto importante para discussão, que é a possibilidade de migração do sistema atual de gestão de cursos, o Tutor MSL, para a plataforma Moodle, de forma integrada à nossa operação.

Nosso questionamento central gira em torno da viabilidade técnica e operacional dessa migração, considerando que atualmente estamos em fase de estruturação e integração dos sistemas — como Spedy, Asaas, WooCommerce e outros — que serão responsáveis por toda a automação de vendas, pagamentos e acesso aos cursos.

É fundamental entender se, uma vez finalizadas essas integrações na estrutura atual, a migração do Tutor MSL para o Moodle poderá ser feita sem gerar prejuízos operacionais, sem perda de dados, de alunos, de matrículas ou de histórico acadêmico, garantindo que todo o ecossistema funcione de forma fluida.

Além disso, precisamos avaliar se, após já estarmos operando comercialmente — vendendo cursos e cadastrando usuários na plataforma atual — essa migração poderá ocorrer de forma transparente, tanto para a empresa quanto para os alunos, sem que haja interrupções nos serviços, na comunicação entre sistemas ou no ambiente de aprendizagem.

Portanto, os pontos principais para esclarecimento são:

Viabilidade técnica da migração do Tutor MSL para o Moodle, considerando todas as integrações já realizadas.

Possibilidade de realizar essa transição sem impacto nas vendas e na experiência dos alunos, especialmente após o início da operação comercial.

Riscos e desafios envolvidos, bem como o que seria necessário para garantir uma migração segura — como backup de dados, replicação de integrações, testes e eventuais ajustes na infraestrutura.

Análise de custo-benefício dessa migração no médio e longo prazo, comparando as funcionalidades, os custos de manutenção, as possibilidades de customização e escalabilidade entre as duas plataformas.




\clearpage

O quadro a seguir apresenta um comparativo entre as plataformas Tutor LMS (plugin para WordPress) e Moodle (plataforma open source independente), considerando aspectos técnicos, operacionais e estratégicos.

\begin{longtable}{|p{3cm}|p{5cm}|p{6cm}|}
	\hline
	\textbf{Critério}                  & \textbf{Tutor LMS (WordPress)}                & \textbf{Moodle (Open Source)}                            \\
	\hline
	\endhead % End of the table header
	\textbf{Natureza}                  & Plugin para WordPress                         & Plataforma independente, open source                     \\
	\hline
	\textbf{Custo inicial}             & Menor; licença anual                          & Gratuito, mas requer infraestrutura e equipe técnica     \\
	\hline
	\textbf{Hospedagem}                & Compartilha com o WordPress                   & Requer hospedagem dedicada (VPS ou cloud)                \\
	\hline
	\textbf{Facilidade de uso}         & Interface amigável para usuários de WordPress & Interface técnica, com curva de aprendizado maior        \\
	\hline
	\textbf{Escalabilidade}            & Limitada à arquitetura do WordPress           & Alta – ideal para expansão institucional                 \\
	\hline
	\textbf{Customização}              & Limitada aos recursos do plugin               & Altíssima – totalmente customizável                      \\
	\hline
	\textbf{Comunidade e suporte}      & Menor, centrada nos criadores do plugin       & Enorme comunidade global e suporte técnico amplo         \\
	\hline
	\textbf{Funcionalidades nativas}   & Funções básicas de LMS                        & Completo: fóruns, gamificação, SCORM, LTI, trilhas, etc. \\
	\hline
	\textbf{Integrações}               & Limitadas ao ecossistema WordPress            & Ampla capacidade de integração via API e plugins         \\
	\hline
	\textbf{Segurança e controle}      & Controle limitado de permissões e perfis      & Controle granular e robusto                              \\
	\hline
	\textbf{Atualizações e manutenção} & Simples, seguindo o ciclo do WordPress        & Requer equipe técnica para manutenções regulares         \\
	\hline
\end{longtable}



Participantes: Sócios do Instituto RBE

\section{1. Abertura e Indagação Central}
Pergunta norteadora da deliberação: “Diante do nosso modelo de negócios, o que desejamos como horizonte estratégico?”

Discussão inicial com os sócios sobre o direcionamento institucional:
\begin{itemize}
	\item Permanecer apenas com cursos livres, com operação enxuta e sem ambição de expansão institucional?
	\item OU - Buscar escalabilidade e crescimento exponencial, com o objetivo de ofertar cursos de pós-graduação, extensão universitária e, futuramente, graduação?
\end{itemize}

\section{2. Contextualização da Situação Atual}
\begin{itemize}
	\item A plataforma inicial selecionada foi o Moodle, entretanto, por falha no controle de e-mails e pagamentos, o ambiente foi descontinuado.
	\item Como solução provisória, adotou-se o Tutor LMS, plataforma mais limitada, vinculada ao WordPress.
	\item No presente momento, a operação técnica (integrações com Spedy, Asaas e WooCommerce) ainda está em fase inicial, o que permite redirecionamento sem perdas significativas.
\end{itemize}

\clearpage
\section{3. Avaliação Estratégica das Alternativas Técnicas}
Comparativo resumido Moodle x Tutor LMS:
\begin{itemize}
	\item O Moodle é mais robusto, escalável e customizável, com funcionalidades avançadas e apropriado para instituições que pretendem operar com cursos de pós e graduação.
	\item O Tutor LMS é limitado, atende bem a operações simples com cursos livres, mas não comporta o crescimento planejado.
\end{itemize}

\section{4. Impactos de uma Migração Posterior (caso permaneçamos com o Tutor LMS)}
\begin{itemize}
	\item Desgaste técnico com reconfiguração de integrações.
	\item Perda de tempo e recursos financeiros e humanos.
	\item Risco de comprometimento da imagem institucional, por intercorrências como:
	      \begin{itemize}
		      \item Perda ou duplicidade de históricos acadêmicos.
		      \item Emissão incorreta de certificados.
		      \item Reclamações de alunos e professores.
	      \end{itemize}
	\item Problemas contábeis e fiscais:
	      \begin{itemize}
		      \item Numeração desordenada de notas fiscais eletrônicas.
		      \item RPS fora de sequência.
		      \item Dificuldade de conciliação contábil.
	      \end{itemize}
\end{itemize}

\section{5. Proposta de Solução Estratégica}
Decisão propositiva: interromper a adoção do Tutor LMS e restabelecer o ambiente Moodle imediatamente, antes da abertura comercial da operação.

Fundamentos da decisão:
\begin{itemize}
	\item O momento atual ainda é favorável para reestruturação.
	\item Evita-se retrabalho futuro e desgaste com usuários.
	\item Adoção do Moodle desde o início permite:
	      \begin{itemize}
		      \item Numeração fiscal sequencial e regular.
		      \item Ambiente institucional robusto e confiável.
		      \item Valorização da imagem da IRB.ACADEMY como referência em educação de qualidade.
	      \end{itemize}
\end{itemize}

\section{6. Governança e Compartilhamento de Acessos}
\begin{itemize}
	\item Proposta de compartilhamento institucionalizado de senhas da hospedagem e do WordPress entre os sócios, mediante uso de cofre digital seguro (ex: 1Password ou Bitwarden).
	\item Definir responsáveis alternados para monitoramento dos alertas de pagamento e prazos contratuais.
	\item Implementar rotina de verificação mensal preventiva da infraestrutura técnica.
\end{itemize}

\clearpage
\section{7. Encaminhamentos e Deliberações}
\begin{itemize}
	\item Votação sobre a adoção definitiva do Moodle.
	\item Indicação de responsável pela reativação da plataforma.
	\item Cronograma para migração das integrações já iniciadas.
	\item Designação de responsável pela gestão de acessos e hospedagem.
	\item Agendamento da próxima reunião de acompanhamento técnico.
\end{itemize}

\section{8. Encerramento}

\includegraphics[width=2.1in]{assiss.png}

Atenciosamente,\\
IRB Academy,\\
Quinta-feira 22/05/2025\\
Possíveis soluções.
\end{document}