\documentclass{article}
\usepackage[utf8]{inputenc}
\usepackage{amsmath} % Required for mathematical environments like align

\title{Momento Linear em Física}
\author{Seu Nome (Opcional)}
\date{\today}

\begin{document}

\maketitle

\begin{abstract}
Este documento apresenta uma explicação concisa sobre o conceito de momento linear em física, abordando sua definição, unidades, relação com a Segunda Lei de Newton e a conservação do momento linear.
\end{abstract}

\section{Introdução ao Momento Linear}
Em física, o \textbf{momento linear}, frequentemente denotado pela letra $p$, é uma grandeza vetorial fundamental que descreve a "quantidade de movimento" de um objeto. Intuitivamente, quanto maior a massa de um objeto e quanto maior sua velocidade, mais difícil será pará-lo ou mudar sua direção. É precisamente isso que o momento linear quantifica.

\section{Definição Matemática}
O momento linear de uma partícula é definido como o produto de sua massa e sua velocidade. Matematicamente, é expresso como:
$$ \vec{p} = m \vec{v} $$
Onde:
\begin{itemize}
    \item $\vec{p}$ é o vetor momento linear.
    \item $m$ é a massa da partícula (uma grandeza escalar).
    \item $\vec{v}$ é o vetor velocidade da partícula.
\end{itemize}
Como a velocidade é uma grandeza vetorial, o momento linear também é uma grandeza vetorial, possuindo magnitude e direção. A direção do vetor momento linear é sempre a mesma direção do vetor velocidade.

\section{Unidades do Momento Linear}
No Sistema Internacional de Unidades (SI), a massa é medida em quilogramas (kg) e a velocidade é medida em metros por segundo (m/s). Portanto, a unidade do momento linear é \textbf{quilograma metro por segundo (kg$\cdot$m/s)}.

\section{Momento Linear e a Segunda Lei de Newton}
A importância do momento linear é ainda mais evidente quando consideramos sua relação com a Segunda Lei de Newton. A formulação original de Newton para sua segunda lei não era $F=ma$, mas sim que a taxa de variação do momento linear de um corpo é proporcional à força resultante que atua sobre ele e ocorre na direção dessa força. Ou seja:
$$ \vec{F}_{\text{resultante}} = \frac{d\vec{p}}{dt} $$
Para um sistema com massa constante, podemos expandir esta equação:
$$ \vec{F}_{\text{resultante}} = \frac{d(m\vec{v})}{dt} = m \frac{d\vec{v}}{dt} = m\vec{a} $$
Onde $\vec{a}$ é a aceleração. Isso demonstra que a conhecida equação $F=ma$ é um caso especial da Segunda Lei de Newton, aplicável quando a massa do objeto é constante.

\section{Conservação do Momento Linear}
Um dos princípios mais importantes em física é o \textbf{Princípio da Conservação do Momento Linear}. Ele afirma que, em um sistema isolado (onde não há forças externas atuando), o momento linear total do sistema permanece constante. Em outras palavras, a soma vetorial dos momentos lineares de todas as partículas dentro do sistema antes de uma interação (como uma colisão) é igual à soma vetorial dos momentos lineares após a interação.
$$ \sum \vec{p}_{\text{inicial}} = \sum \vec{p}_{\text{final}} $$
Este princípio é fundamental para analisar colisões, explosões e outros eventos onde as forças internas são muito maiores que quaisquer forças externas.

\section{Conclusão}
O momento linear é um conceito central na física, fornecendo uma medida quantitativa da "quantidade de movimento" de um objeto. Sua definição, relação com as Leis de Newton e o princípio da conservação o tornam uma ferramenta indispensável para a análise de diversos fenômenos físicos, desde o movimento de partículas subatômicas até colisões de corpos celestes.

\end{document}