\documentclass{article}
\usepackage[utf8]{inputenc}
\usepackage{amsmath} % Para ambientes matemáticos como align

\title{As Leis de Kepler do Movimento Planetário}
\author{Seu Nome/Nome da Instituição}
\date{\today}

\begin{document}

\maketitle

\begin{abstract}
Este documento apresenta uma explicação detalhada das três leis de Kepler do movimento planetário, que descrevem como os planetas orbitam o Sol. As leis foram formuladas por Johannes Kepler no início do século XVII e foram fundamentais para o desenvolvimento posterior da lei da gravitação universal de Isaac Newton.
\end{abstract}

\section{Introdução}
As Leis de Kepler são três leis científicas que descrevem o movimento dos planetas em torno do Sol. Elas foram deduzidas pelo astrônomo e matemático alemão Johannes Kepler (1571-1630) a partir de observações astronômicas detalhadas feitas por Tycho Brahe (1546-1601). Publicadas entre 1609 e 1619, essas leis marcaram uma transição significativa da visão geocêntrica para a heliocêntrica do sistema solar e pavimentaram o caminho para a lei da gravitação universal de Isaac Newton.

\section{A Primeira Lei de Kepler: Lei das Órbitas Elípticas}
A primeira lei de Kepler descreve a forma das órbitas planetárias. Ela afirma que:

\begin{quote}
\textit{A órbita de cada planeta é uma elipse, com o Sol em um dos focos.}
\end{quote}

Matematicamente, uma elipse pode ser descrita por sua excentricidade (e), que determina o quão "achatada" ela é, e pelo seu semieixo maior (a). Para uma órbita planetária, o Sol não está no centro da elipse, mas sim em um dos dois pontos focais. Isso significa que a distância entre o planeta e o Sol varia ao longo da órbita.

\section{A Segunda Lei de Kepler: Lei das Áreas Iguais}
A segunda lei de Kepler aborda a velocidade com que um planeta se move em sua órbita. Ela estabelece que:

\begin{quote}
\textit{Uma linha imaginária que liga o Sol a um planeta varre áreas iguais em intervalos de tempo iguais.}
\end{quote}

Esta lei implica que um planeta se move mais rapidamente quando está mais próximo do Sol (no periélio) e mais lentamente quando está mais distante (no afélio). Embora a velocidade orbital varie, a taxa na qual a área é varrida é constante. Matematicamente, se A é a área varrida e t é o tempo, então:

dt
dA
​
 =constante
Este princípio é uma consequência direta da conservação do momento angular.

\section{A Terceira Lei de Kepler: Lei dos Períodos}
A terceira lei de Kepler relaciona o período orbital de um planeta com o tamanho de sua órbita. Ela declara que:

\begin{quote}
\textit{O quadrado do período orbital de um planeta é diretamente proporcional ao cubo do semieixo maior de sua órbita.}
\end{quote}

Em outras palavras, quanto maior a órbita de um planeta, mais tempo ele leva para completá-la. A relação pode ser expressa como:

T 
2
 ∝a 
3
 
Onde T é o período orbital do planeta e a é o comprimento do semieixo maior de sua órbita elíptica. Para todos os planetas orbitando o mesmo corpo central (neste caso, o Sol), a constante de proporcionalidade é a mesma:

a 
3
 
T 
2
 
​
 =K
Onde K é a constante de Kepler, que depende apenas da massa do corpo central (o Sol) e da constante gravitacional universal G. Esta lei foi posteriormente explicada por Newton como uma consequência da lei da gravitação universal.

\section{Conclusão}
As Leis de Kepler representam um marco fundamental na história da astronomia e da física. Elas não só descreveram com precisão o movimento planetário, mas também forneceram a base empírica para a formulação da lei da gravitação universal de Newton, unificando os fenômenos celestes e terrestres sob uma única estrutura teórica. A simplicidade e a elegância dessas leis continuam a fascinar e a inspirar cientistas até hoje.

\begin{thebibliography}{9}
\bibitem{Kepler1609} Kepler, J. (1609). \textit{Astronomia nova...}. Prague.
\bibitem{Kepler1619} Kepler, J. (1619). \textit{Harmonices Mundi}. Linz.
\bibitem{Newton1687} Newton, I. (1687). \textit{Philosophiæ Naturalis Principia Mathematica}. London.
\end{thebibliography}

\end{document}