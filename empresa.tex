\documentclass[11pt, a4paper]{article}


% --- PACOTES GERAIS ---
\usepackage[utf8]{inputenc} % Codificação de entrada
\usepackage[T1]{fontenc}    % Codificação de fonte
\usepackage[brazilian]{babel} % Idioma português do Brasil
\usepackage{geometry}       % Controle de margens
\usepackage{fancyhdr}       % Para cabeçalhos e rodapés personalizados
\usepackage{graphicx}       % Para incluir logotipos ou imagens
\usepackage{xcolor}         % Para cores personalizadas
\usepackage{lipsum}         % Para texto de exemplo (lorem ipsum)

% --- CONFIGURAÇÃO DE GEOMETRIA (MARGENS) ---
\geometry{
    a4paper,
    top=2.5cm,
    bottom=2.5cm,
    left=3cm,
    right=3cm,
    headheight=14pt,
    footskip=1.2cm
}

% --- CORES PERSONALIZADAS ---
\definecolor{corporateblue}{HTML}{005691} % Azul corporativo
\definecolor{corporategray}{HTML}{505050} % Cinza corporativo

% --- CONFIGURAÇÃO DO CABEÇALHO E RODAPÉ ---
\pagestyle{fancy}
\fancyhf{} % Limpa cabeçalhos e rodapés padrão

% Cabeçalho: Nome da Empresa Centralizado
\fancyhead[C]{%
    \textsf{\color{corporateblue}\textbf{\Large IRB Academy
    }}
}

% Rodapé: Número da página (opcional, mas bom para documentos longos)
\fancyfoot[R]{%
    \textsf{\color{corporategray}\thepage}
}

\renewcommand{\headrulewidth}{0.8pt} % Linha abaixo do cabeçalho
\renewcommand{\footrulewidth}{0pt}   % Remove linha no rodapé (se não quiser)

% --- ESTILOS DE TÍTULO ---
\newcommand{\documenttitle}[1]{%
    \vspace*{1.5cm} % Espaço antes do título
    \begin{center}
        \textsf{\color{corporateblue}\bfseries\Huge #1}
    \end{center}
    \vspace*{1cm} % Espaço depois do título
}

% --- INÍCIO DO DOCUMENTO ---
\begin{document}

% --- TÍTULO DO DOCUMENTO ---
\documenttitle{Estratégias de Marketing}

% --- CORPO DO TEXTO ---
\section*{Introdução} % Seção sem numeração
\textsf{\color{corporategray} A estratégia de marketing proposta contempla uma abordagem multicanal, visando ampliar a visibilidade da empresa e atrair um público qualificado. Entre as principais ações estão a elaboração de um workshop gratuito no YouTube, que proporcionará uma experiência prática e educativa aos participantes, e a produção de um vídeo criativo voltado para campanhas no Facebook e Instagram, com foco em gerar engajamento e alcance nas redes sociais. Além disso, será desenvolvido um curso gratuito no YouTube, com conteúdo introdutório sobre o tema proposto pelo professor, com o objetivo de gerar autoridade e atrair potenciais clientes. A estratégia também inclui campanhas no Google Ads, visando atrair usuários que buscam ativamente pelo tema, e anúncios no YouTube, que potencializarão o alcance dos conteúdos audiovisuais da empresa.} % Texto de exemplo

\vspace{0.5cm} % Espaço

\section*{Análise de Resultados}
\textsf{\color{corporategray} A análise de resultados será realizada por meio da elaboração de relatórios de desempenho específicos para cada campanha, utilizando as métricas disponibilizadas pelas plataformas YouTube, Facebook, Instagram e Google. Esses relatórios permitirão avaliar indicadores como alcance, visualizações, taxa de cliques (CTR), engajamento, conversões e custo por resultado. A partir desses dados, será possível construir um parâmetro de eficiência entre as campanhas, identificando quais ações geraram maior retorno e quais pontos podem ser otimizados. Essa avaliação contínua será fundamental para ajustar as estratégias, maximizar os investimentos e aprimorar os resultados ao longo do tempo.} % Texto de exemplo

\subsection*{Vendas}
\textsf{\color{corporategray} As estratégias de vendas para cursos, aliadas a um suporte ao cliente eficiente e a uma prospecção ativa de leads nas redes sociais, são fundamentais para o crescimento sustentável do negócio. Através das redes sociais, é possível não apenas divulgar os cursos, mas também atrair potenciais alunos de forma segmentada, gerando relacionamentos e construindo autoridade no mercado. A prospecção de leads qualificados permite que a empresa se conecte diretamente com pessoas interessadas no tema, aumentando significativamente as chances de conversão. Além disso, oferecer um suporte ágil e eficiente durante todo o processo de compra e pós-venda é essencial para garantir a satisfação dos alunos, fortalecer a reputação da marca e estimular recomendações espontâneas, criando assim um ciclo contínuo de atração, conversão e fidelização de clientes.} % Texto de exemplo

\subsection*{Marketing}
\textsf{\color{corporategray} Dentro das ações de marketing, a produção de um vídeo no formato vertical, com resolução de 1080x1920 pixels, será uma peça fundamental para as campanhas nas redes sociais, especialmente em plataformas como Instagram, Facebook e YouTube Shorts. Nos primeiros cinco segundos do vídeo será inserida uma mensagem de impacto, capaz de prender a atenção do lead e gerar interesse imediato, destacando de forma clara e objetiva as principais vantagens de adquirir o curso, como benefícios, diferenciais e transformação que ele oferece. Na sequência, o vídeo apresentará um conteúdo informativo, onde serão abordados pontos relevantes sobre o tema do curso, demonstrando a autoridade do professor e a qualidade do conteúdo. Por fim, o vídeo encerrará com uma chamada para ação (CTA) direta e persuasiva, incentivando o lead a realizar a inscrição, acessar mais informações ou efetuar a compra imediatamente.} % Texto de exemplo

\section*{Conclusão}
\textsf{\color{corporategray} Diante de um mercado cada vez mais competitivo, uma boa tomada de decisão aliada a estratégias bem definidas de marketing torna-se essencial para o sucesso na captação de leads. Planejar cada etapa, desde a criação de conteúdos relevantes até a execução de campanhas bem segmentadas, impacta diretamente na atração do público certo e no aumento das conversões. Investir em estratégias inteligentes não só potencializa os resultados, como também otimiza recursos, reduz custos e fortalece a presença da marca no ambiente digital. Portanto, adotar uma abordagem estratégica, baseada em dados, testes e análises constantes, é indispensável para garantir um fluxo contínuo de novos leads e, consequentemente, o crescimento sustentável do negócio.} % Texto de exemplo

\vspace{1cm} % Espaço antes da assinatura
\begin{flushright}
	\textsf{\color{corporategray}Atenciosamente,}\\
	\textsf{\color{corporategray}Luiz Henrique Lamim C. Brito}\\
	\textsf{\color{corporategray}Data: \today}
\end{flushright}

\end{document}