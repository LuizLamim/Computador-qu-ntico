\documentclass{article}
\usepackage[utf8]{inputenc}
\usepackage{amsmath} % Para equações e comandos matemáticos
\usepackage{amsfonts}
\usepackage{amssymb}
\usepackage[brazil]{babel} % Para hifenização e outras configurações em português

\title{As Leis do Movimento de Newton}
\author{} % Você pode adicionar seu nome aqui
\date{} % Você pode colocar a data ou deixar em branco para a data atual

\begin{document}

\maketitle

As principais "equações" de Newton são, na verdade, as suas \textbf{Três Leis do Movimento}, que formam a base da mecânica clássica. Embora a primeira e a terceira leis sejam mais conceituais, a segunda lei é expressa por uma equação fundamental:

---

\section{1ª Lei de Newton: Lei da Inércia}

Esta lei descreve o comportamento de um corpo quando a \textbf{força resultante} sobre ele é nula.

\begin{itemize}
    \item \textbf{Enunciado:} "Todo corpo persiste em seu estado de repouso, ou de movimento retilíneo uniforme, a menos que seja compelido a modificar esse estado pela ação de forças impressas sobre ele."
    \item \textbf{Conceito:} Em termos mais simples, um objeto parado tende a permanecer parado, e um objeto em movimento em linha reta com velocidade constante tende a permanecer em movimento com a mesma velocidade e direção, a menos que uma força externa atue sobre ele.
    \item \textbf{Matematicamente (implícito):} Se a \textbf{força resultante} ($\vec{F}_{R}$) é igual a zero, então a \textbf{velocidade} ($\vec{v}$) é constante (o que inclui o repouso, onde $\vec{v} = 0$).
    $$ \vec{F}_{R} = 0 \implies \vec{v} = \text{constante} $$
\end{itemize}

---

\section{2ª Lei de Newton: Princípio Fundamental da Dinâmica}

Esta é a lei mais quantitativa e é expressa por uma equação. Ela relaciona a força, a massa e a aceleração de um corpo.

\begin{itemize}
    \item \textbf{Enunciado:} "A mudança de movimento é proporcional à força motora imprimida, e é produzida na direção de linha reta na qual aquela força é aplicada."
    \item \textbf{Conceito:} A força resultante que age sobre um corpo é diretamente proporcional à sua aceleração e tem a mesma direção e sentido da aceleração. A massa do corpo é a constante de proporcionalidade.
    \item \textbf{Matematicamente:}
    $$ \vec{F}_{R} = m \cdot \vec{a} $$
    Onde:
    \begin{itemize}
        \item $\vec{F}_{R}$ é a \textbf{força resultante} (em Newtons, N). É a soma vetorial de todas as forças que atuam sobre o corpo.
        \item $m$ é a \textbf{massa} do corpo (em quilogramas, kg).
        \item $\vec{a}$ é a \textbf{aceleração} do corpo (em metros por segundo ao quadrado, m/s$^2$).
    \end{itemize}
\end{itemize}

---

\section{3ª Lei de Newton: Lei da Ação e Reação}

Esta lei descreve a interação entre dois corpos.

\begin{itemize}
    \item \textbf{Enunciado:} "A toda ação há sempre uma reação oposta e de igual intensidade: as ações mútuas de dois corpos um sobre o outro são sempre iguais e dirigidas em sentidos opostos."
    \item \textbf{Conceito:} Se um corpo A exerce uma força sobre um corpo B (ação), então o corpo B exercerá uma força de mesma magnitude e direção, mas em sentido oposto, sobre o corpo A (reação). As forças de ação e reação atuam sempre em pares e em corpos diferentes.
    \item \textbf{Matematicamente (implícito):} Se $\vec{F}_{AB}$ é a força exercida por A em B, e $\vec{F}_{BA}$ é a força exercida por B em A, então:
    $$ \vec{F}_{AB} = - \vec{F}_{BA} $$
\end{itemize}

Essas três leis são a base da compreensão do movimento e das interações de objetos no universo macroscópico e em velocidades muito menores que a da luz.

\end{document}