\documentclass[a4paper,12pt]{article}
\usepackage[brazil]{babel}
\usepackage[utf8]{inputenc}
\usepackage[T1]{fontenc}
\usepackage{amsmath}
\usepackage{amssymb}
\usepackage{geometry}
\usepackage{fancyhdr}
\usepackage{tcolorbox}

% Configuração das Margens
\geometry{top=2.5cm, bottom=2.5cm, left=3cm, right=3cm}

% Título e Autor
\title{\textbf{Resumo de Análise Combinatória}}
\author{Guia de Estudo}
\date{\today}

\begin{document}

\maketitle
\thispagestyle{fancy}

\section{Introdução}
A \textbf{Análise Combinatória} é a área da matemática que estuda métodos de contagem. O objetivo é determinar o número de possibilidades de ocorrência de um evento sem a necessidade de enumerar cada caso individualmente.

\section{Conceitos Fundamentais}

\subsection{Princípio Fundamental da Contagem (PFC)}
Também conhecido como Princípio Multiplicativo. Se um evento é composto por $k$ etapas independentes e sucessivas, onde:
\begin{itemize}
    \item A 1ª etapa pode ocorrer de $n_1$ maneiras;
    \item A 2ª etapa pode ocorrer de $n_2$ maneiras;
    \item ...
    \item A $k$-ésima etapa pode ocorrer de $n_k$ maneiras.
\end{itemize}
Então, o número total de possibilidades é dado pelo produto:
\begin{equation}
    Total = n_1 \times n_2 \times \dots \times n_k
\end{equation}

\subsection{Fatorial}
O fatorial de um número natural $n$, denotado por $n!$, é o produto de todos os inteiros positivos menores ou iguais a $n$.
\begin{equation}
    n! = n \times (n-1) \times (n-2) \times \dots \times 1
\end{equation}
\textbf{Definição importante:} $0! = 1$.

\section{Agrupamentos}
A principal distinção entre os tipos de agrupamento reside na resposta à pergunta: \textit{A ordem dos elementos importa?}

\subsection{Permutações}
Usado quando organizamos \textbf{todos} os $n$ elementos disponíveis. A ordem importa.

\subsubsection*{Permutação Simples}
Arranjo de $n$ elementos distintos.
\begin{equation}
    P_n = n!
\end{equation}

\subsubsection*{Permutação com Repetição}
Quando alguns elementos do conjunto são iguais. Se temos $n$ elementos, onde um repete $\alpha$ vezes, outro $\beta$ vezes, etc.:
\begin{equation}
    P_n^{\alpha, \beta, \dots} = \frac{n!}{\alpha! \cdot \beta! \cdot \dots}
\end{equation}

\subsection{Arranjos Simples}
Usado quando selecionamos $p$ elementos de um grupo de $n$ disponíveis ($p \le n$) e a \textbf{ordem importa} (ex: pódio de corrida, senhas).
\begin{equation}
    A_{n,p} = \frac{n!}{(n-p)!}
\end{equation}

\subsection{Combinações Simples}
Usado quando selecionamos $p$ elementos de um grupo de $n$ disponíveis ($p \le n$) e a \textbf{ordem NÃO importa} (ex: sorteio de loteria, formar equipes).
\begin{equation}
    C_{n,p} = \binom{n}{p} = \frac{n!}{p!(n-p)!}
\end{equation}

\section{Quadro Resumo}

\begin{center}
\begin{tcolorbox}[colback=gray!10, colframe=black, title=Como escolher a fórmula?]
\begin{enumerate}
    \item \textbf{Estou usando todos os elementos?}
    \begin{itemize}
        \item Sim $\rightarrow$ \textbf{Permutação} ($P_n$ ou $P_n^{\alpha \dots}$).
        \item Não (apenas uma parte) $\rightarrow$ Vá para a pergunta 2.
    \end{itemize}
    \item \textbf{A ordem importa?} (Trocar a posição muda o resultado?)
    \begin{itemize}
        \item Sim $\rightarrow$ \textbf{Arranjo} ($A_{n,p}$).
        \item Não $\rightarrow$ \textbf{Combinação} ($C_{n,p}$).
    \end{itemize}
\end{enumerate}
\end{tcolorbox}
\end{center}

\end{document}