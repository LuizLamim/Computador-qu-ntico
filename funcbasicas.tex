\documentclass{article}
\usepackage[utf8]{inputenc}
\usepackage{amsmath} % Para comandos matemáticos como \texttt
\usepackage{amsfonts}
\usepackage{amssymb}
\usepackage[brazil]{babel} % Para hifenização e outras configurações em português

\title{Funções Básicas em Assembly}
\author{} % Você pode adicionar seu nome aqui se desejar
\date{} % Você pode colocar a data ou deixar em branco para a data atual

\begin{document}

\maketitle

\section{Funções Básicas em Assembly}

Assembly é uma linguagem de programação de baixo nível que interage diretamente com a arquitetura do processador. É frequentemente utilizada para tarefas que exigem alto desempenho, controle preciso sobre o hardware, ou em sistemas embarcados. A seguir, apresentamos algumas funções básicas e conceitos fundamentais.

---

\subsection{Registradores}

Os \textbf{registradores} são pequenas áreas de armazenamento de dados dentro da CPU, usados para operações rápidas. Alguns dos registradores mais comuns na arquitetura x86 incluem:

\begin{itemize}
    \item \texttt{EAX}, \texttt{EBX}, \texttt{ECX}, \texttt{EDX}: Registradores de propósito geral, frequentemente usados para armazenar dados e resultados de cálculos.
    \item \texttt{ESI}, \texttt{EDI}: Usados como registradores de índice para operações de memória, como cópia de blocos de dados.
    \item \texttt{EBP}, \texttt{ESP}: Respectivamente, o \textbf{ponteiro de base} (base pointer) e o \textbf{ponteiro de pilha} (stack pointer), cruciais para o gerenciamento da pilha de chamadas de função.
    \item \texttt{EIP}: O \textbf{ponteiro de instrução} (instruction pointer), que armazena o endereço da próxima instrução a ser executada.
\end{itemize}

---

\subsection{Movimentação de Dados (\texttt{MOV})}

A instrução $\texttt{MOV}$ é uma das mais fundamentais, utilizada para copiar dados entre registradores, entre registradores e memória, ou para carregar valores imediatos em registradores.

Sintaxe geral: \texttt{MOV destino, origem}

Exemplos:

\begin{itemize}
    \item \texttt{MOV EAX, 10}: Move o valor imediato 10 para o registrador \texttt{EAX}.
    \item \texttt{MOV EBX, EAX}: Copia o conteúdo de \texttt{EAX} para \texttt{EBX}.
    \item \texttt{MOV [minha\_variavel], EAX}: Copia o conteúdo de \texttt{EAX} para o endereço de memória rotulado como \texttt{minha\_variavel}.
    \item \texttt{MOV EAX, [minha\_variavel]}: Copia o conteúdo da memória em \texttt{minha\_variavel} para \texttt{EAX}.
\end{itemize}

---

\subsection{Operações Aritméticas}

Assembly oferece instruções para as operações aritméticas básicas.

\subsubsection{Adição (\texttt{ADD})}

A instrução $\texttt{ADD}$ realiza a soma de dois operandos e armazena o resultado no primeiro operando.

Sintaxe: \texttt{ADD destino, origem}

Exemplo:

\begin{itemize}
    \item \texttt{ADD EAX, EBX}: Soma o conteúdo de \texttt{EBX} a \texttt{EAX}, e o resultado é armazenado em \texttt{EAX}.
\end{itemize}

\subsubsection{Subtração (\texttt{SUB})}

A instrução $\texttt{SUB}$ realiza a subtração do segundo operando do primeiro, armazenando o resultado no primeiro.

Sintaxe: \texttt{SUB destino, origem}

Exemplo:

\begin{itemize}
    \item \texttt{SUB EAX, 5}: Subtrai 5 de \texttt{EAX}, e o resultado é armazenado em \texttt{EAX}.
\end{itemize}

\subsubsection{Multiplicação (\texttt{MUL} e \texttt{IMUL})}

$\texttt{MUL}$ (multiplicação sem sinal) e $\texttt{IMUL}$ (multiplicação com sinal) são usadas para multiplicar. A forma de uso pode variar dependendo do tamanho dos operandos.

Exemplo (multiplicação de 32 bits por 32 bits em \texttt{EAX}):

\begin{verbatim}
MOV EAX, 10
MOV EBX, 5
IMUL EBX ; Multiplica EAX por EBX. O resultado (50) é armazenado em EAX (se couber) ou EDX:EAX para resultados maiores.
\end{verbatim}

\subsubsection{Divisão (\texttt{DIV} e \texttt{IDIV})}

$\texttt{DIV}$ (divisão sem sinal) e $\texttt{IDIV}$ (divisão com sinal) são usadas para dividir.

Exemplo (divisão de 32 bits em \texttt{EAX} por um registrador):

\begin{verbatim}
MOV EAX, 100
MOV EBX, 10
CDQ ; Estende o sinal de EAX para EDX para prepará-lo para a divisão de 64 bits por 32 bits.
IDIV EBX ; Divide EDX:EAX por EBX. O quociente (10) é armazenado em EAX e o resto (0) em EDX.
\end{verbatim}

---

\subsection{Comparação e Salto Condicional}

As instruções de comparação e salto são essenciais para controlar o fluxo do programa.

\subsubsection{Comparação (\texttt{CMP})}

A instrução $\texttt{CMP}$ compara dois operandos, definindo \textit{flags} no registrador de \textit{flags} (como ZF - Zero Flag, CF - Carry Flag, SF - Sign Flag) que podem ser testados por instruções de salto condicional.

Sintaxe: \texttt{CMP operando1, operando2}

Exemplo:

\begin{itemize}
    \item \texttt{CMP EAX, EBX}: Compara o conteúdo de \texttt{EAX} com \texttt{EBX}.
\end{itemize}

\subsubsection{Saltos Condicionais (\texttt{JGE}, \texttt{JE}, \texttt{JNE}, etc.)}

Instruções de salto condicional alteram o fluxo de execução com base no estado dos \textit{flags} após uma comparação.

Exemplos:

\begin{itemize}
    \item \texttt{JE label}: Salta para \texttt{label} se os operandos forem \textbf{iguais} (Zero Flag = 1).
    \item \texttt{JNE label}: Salta para \texttt{label} se os operandos forem \textbf{diferentes} (Zero Flag = 0).
    \item \texttt{JG label}: Salta para \texttt{label} se o primeiro operando for \textbf{maior} que o segundo (com sinal).
    \item \texttt{JGE label}: Salta para \texttt{label} se o primeiro operando for \textbf{maior ou igual} ao segundo (com sinal).
    \item \texttt{JL label}: Salta para \texttt{label} se o primeiro operando for \textbf{menor} que o segundo (com sinal).
    \item \texttt{JLE label}: Salta para \texttt{label} se o primeiro operando for \textbf{menor ou igual} ao segundo (com sinal).
\end{itemize}

Exemplo de uso:

\begin{verbatim}
    CMP EAX, EBX
    JG maior_que_EBX
    ; Código a ser executado se EAX <= EBX
    JMP fim_do_bloco

maior_que_EBX:
    ; Código a ser executado se EAX > EBX

fim_do_bloco:
    ; Continuação do programa
\end{verbatim}

---

\subsection{Pilha (\texttt{PUSH} e \texttt{POP})}

A \textbf{pilha} é uma estrutura de dados LIFO (Last-In, First-Out) usada para armazenar temporariamente dados, como endereços de retorno de funções e parâmetros.

\begin{itemize}
    \item \texttt{PUSH}: Coloca um valor no topo da pilha, decrementando o \texttt{ESP}.
    \item \texttt{POP}: Remove um valor do topo da pilha, incrementando o \texttt{ESP}.
\end{itemize}

Exemplos:

\begin{itemize}
    \item \texttt{PUSH EAX}: Coloca o conteúdo de \texttt{EAX} na pilha.
    \item \texttt{POP EBX}: Remove o valor do topo da pilha e o armazena em \texttt{EBX}.
\end{itemize}

---

\subsection{Chamada e Retorno de Função (\texttt{CALL} e \texttt{RET})}

As instruções $\texttt{CALL}$ e $\texttt{RET}$ são usadas para gerenciar chamadas de sub-rotinas (funções).

\begin{itemize}
    \item \texttt{CALL nome\_da\_funcao}: Empurra o endereço da próxima instrução para a pilha e salta para \texttt{nome\_da\_funcao}.
    \item \texttt{RET}: Remove o endereço de retorno da pilha e salta para ele, retornando da função.
\end{itemize}

Exemplo:

\begin{verbatim}
; Função de exemplo
minha_funcao:
    ; Código da função
    RET

; Bloco principal
    CALL minha_funcao
    ; Código após o retorno da função
\end{verbatim}

Este texto fornece uma introdução básica às funções de assembly. O domínio completo dessa linguagem requer um estudo aprofundado da arquitetura do processador e das convenções de chamada de sistema.

\end{document}