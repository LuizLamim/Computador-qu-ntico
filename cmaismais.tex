\documentclass{article}
\usepackage[utf8]{inputenc}
\usepackage[portuguese]{babel}
\usepackage{amsmath} % Para equações e símbolos matemáticos (se necessário)
\usepackage{amsfonts} % Para fontes matemáticas (se necessário)
\usepackage{amssymb} % Para símbolos matemáticos adicionais (se necessário)
\usepackage{graphicx} % Para incluir imagens (se necessário)
\usepackage{listings} % Para formatar blocos de código
\usepackage{xcolor} % Para cores no listings

% Configuração para o listings (para blocos de código C++)
\definecolor{codegray}{rgb}{0.5,0.5,0.5}
\definecolor{codepurple}{rgb}{0.58,0,0.82}
\definecolor{backcolour}{rgb}{0.98,0.98,0.98}

\lstset{
    backgroundcolor=\color{backcolour},
    commentstyle=\color{codegray},
    keywordstyle=\color{blue},
    numberstyle=\tiny\color{codegray},
    stringstyle=\color{codepurple},
    basicstyle=\ttfamily\footnotesize,
    breakatwhitespace=false,
    breaklines=true,
    captionpos=b,
    keepspaces=true,
    numbers=left,
    numbersep=5pt,
    showspaces=false,
    showstringspaces=false,
    showtabs=false,
    tabsize=2,
    language=C++, % Linguagem do código
    morekeywords={std,cout,cin,endl,if,else,for,while,int,float,double,char,bool,void,return,include,main},
    identifierstyle=\color{black}
}

\title{Noções Básicas de Programação em C++}
\author{Seu Nome / Autor Desconhecido}
\date{\today}

\begin{document}

\maketitle

\begin{abstract}
Este documento apresenta uma introdução às funções e conceitos básicos da linguagem de programação C++. Ele aborda a estrutura fundamental de um programa, tipos de dados, variáveis, operadores, entrada e saída, e estruturas de controle de fluxo, fornecendo exemplos práticos para facilitar a compreensão.
\end{abstract}

\section{Introdução ao C++}
C++ é uma linguagem de programação de propósito geral, orientada a objetos e multi-paradigma, derivada da linguagem C. É amplamente utilizada em desenvolvimento de sistemas, jogos, aplicações de alto desempenho e sistemas embarcados devido à sua performance e flexibilidade.

\section{Estrutura Básica de um Programa C++}
Todo programa C++ possui uma estrutura fundamental. Abaixo, um exemplo simples:

\begin{lstlisting}[caption={Estrutura Básica}]
#include <iostream> // Inclui a biblioteca de entrada/saída

int main() { // Função principal, ponto de entrada do programa
    std::cout << "Olá, Mundo!" << std::endl; // Imprime texto na tela
    return 0; // Indica que o programa terminou com sucesso
}
\end{lstlisting}

\begin{itemize}
    \item \textbf{\texttt{\#include <iostream>}}: Esta linha é uma diretiva de pré-processador que inclui a biblioteca \texttt{iostream}. Ela fornece as funcionalidades para operações de entrada e saída, como imprimir dados no console (\texttt{std::cout}) e ler dados do teclado (\texttt{std::cin}).
    \item \textbf{\texttt{int main()}}: Esta é a função principal do programa. A execução de qualquer programa C++ sempre começa a partir desta função. O \texttt{int} indica que a função retorna um valor inteiro (geralmente \texttt{0} para sucesso).
    \item \textbf{\texttt{std::cout << "Olá, Mundo!" << std::endl;}}: Esta linha é responsável por imprimir a frase "Olá, Mundo!" no console.
    \begin{itemize}
        \item \texttt{std::cout}: Objeto usado para saída padrão (console).
        \item \texttt{<<}: Operador de inserção, usado para enviar dados para \texttt{std::cout}.
        \item \texttt{"Olá, Mundo!"}: A string de texto a ser impressa.
        \item \texttt{std::endl}: Insere uma nova linha e limpa o buffer de saída.
    \end{itemize}
    \item \textbf{\texttt{return 0;}}: Esta linha indica que a função \texttt{main} terminou sua execução e retornou um valor \texttt{0} para o sistema operacional, que geralmente significa que o programa foi executado com sucesso.
\end{itemize}

\section{Variáveis e Tipos de Dados}
\textbf{Variáveis} são "contêineres" para armazenar dados. Cada variável possui um \textbf{tipo de dado} que define o tipo de informação que ela pode conter (números inteiros, números decimais, caracteres, etc.) e a quantidade de memória que ela ocupará.

Alguns tipos de dados básicos em C++ incluem:
\begin{itemize}
    \item \textbf{\texttt{int}}: Armazena números inteiros (ex: -5, 0, 100).
    \item \textbf{\texttt{float}}: Armazena números de ponto flutuante de precisão simples (ex: 3.14, -0.5).
    \item \textbf{\texttt{double}}: Armazena números de ponto flutuante de precisão dupla (mais preciso que \texttt{float}).
    \item \textbf{\texttt{char}}: Armazena um único caractere (ex: 'a', 'Z', '5').
    \item \textbf{\texttt{bool}}: Armazena valores booleanos (\texttt{true} ou \texttt{false}).
\end{itemize}

\begin{lstlisting}[caption={Exemplo de Declaração de Variáveis}]
int idade = 30;
double preco = 19.99;
char inicial = 'J';
bool estaAtivo = true;
\end{lstlisting}

\section{Operadores Básicos}
Operadores são símbolos que realizam operações em variáveis e valores.
\begin{itemize}
    \item \textbf{Aritméticos}: \texttt{+}, \texttt{-}, \texttt{*}, \texttt{/}, \texttt{\%} (módulo).
    \item \textbf{Atribuição}: \texttt{=} (atribui um valor), \texttt{+=}, \texttt{-=}, \texttt{*=}, \texttt{/=}.
    \item \textbf{Comparação}: \texttt{==} (igual a), \texttt{!=} (diferente de), \texttt{<}, \texttt{>}, \texttt{<=}, \texttt{>=}.
    \item \textbf{Lógicos}: \texttt{&&} (E lógico), \texttt{||} (OU lógico), \texttt{!} (NÃO lógico).
\end{itemize}

\begin{lstlisting}[caption={Exemplo de Operadores}]
int a = 10, b = 3;
int soma = a + b; // soma = 13
int resto = a % b; // resto = 1
bool maior = (a > b); // maior = true
\end{lstlisting}

\section{Entrada e Saída (Input/Output)}
Além do \texttt{std::cout} para saída, usamos \texttt{std::cin} para entrada de dados do usuário.

\begin{lstlisting}[caption={Exemplo de Entrada/Saída}]
#include <iostream>

int main() {
    int numero;
    std::cout << "Digite um número: ";
    std::cin >> numero; // Lê o número digitado pelo usuário
    std::cout << "Você digitou: " << numero << std::endl;
    return 0;
}
\end{lstlisting}
\begin{itemize}
    \item \texttt{std::cin}: Objeto usado para entrada padrão (teclado).
    \item \texttt{>>}: Operador de extração, usado para obter dados de \texttt{std::cin} e armazená-los em uma variável.
\end{itemize}

\section{Estruturas de Controle de Fluxo}
As estruturas de controle permitem que você defina a ordem em que as instruções são executadas.

\subsection{Condicionais (\texttt{if}, \texttt{else if}, \texttt{else})}
Executam blocos de código diferentes com base em uma condição.

\begin{lstlisting}[caption={Exemplo de Condicional}]
int idade = 18;
if (idade >= 18) {
    std::cout << "Você é maior de idade." << std::endl;
} else {
    std::cout << "Você é menor de idade." << std::endl;
}
\end{lstlisting}

\subsection{Loops (\texttt{for}, \texttt{while})}
Repetem um bloco de código várias vezes.

\subsubsection{\texttt{for} Loop}
Ideal quando o número de iterações é conhecido.

\begin{lstlisting}[caption={\texttt{for} Loop}]
// Imprime números de 0 a 4
for (int i = 0; i < 5; ++i) {
    std::cout << i << " ";
}
std::cout << std::endl; // Saída: 0 1 2 3 4
\end{lstlisting}

\subsubsection{\texttt{while} Loop}
Ideal quando o número de iterações é desconhecido e o loop continua enquanto uma condição é verdadeira.

\begin{lstlisting}[caption={\texttt{while} Loop}]
int contador = 0;
while (contador < 3) {
    std::cout << "Contador: " << contador << std::endl;
    contador++;
}
\end{lstlisting}

\section{Conclusão}
Este documento forneceu uma visão geral das funções básicas e da sintaxe do C++. Dominar esses conceitos fundamentais é o primeiro passo crucial para construir programas mais complexos e explorar os recursos avançados que C++ oferece. A prática contínua é essencial para solidificar o aprendizado.

\end{document}