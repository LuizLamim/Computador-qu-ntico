\documentclass{article}
\usepackage[utf8]{inputenc}
\usepackage[portuguese]{babel}
\usepackage{amsmath} % Para equações matemáticas

\title{As Órbitas Planetárias: Uma Dança Cósmica da Gravidade}
\author{Seu Nome (ou O Gerador de Texto)}
\date{\today}

\begin{document}

\maketitle

\begin{abstract}
Este documento explora os princípios fundamentais das órbitas planetárias, abordando as leis de Kepler e a lei da gravitação universal de Newton. Compreenderemos como esses conceitos se entrelaçam para descrever o movimento dos planetas ao redor do Sol.
\end{abstract}

\section{Introdução}
O estudo do movimento dos corpos celestes tem fascinado a humanidade por milênios. Desde as observações antigas até as sofisticadas teorias modernas, a compreensão das órbitas planetárias representa um marco na nossa capacidade de decifrar o universo. Este texto visa apresentar os conceitos-chave que governam a fascinante dança dos planetas.

\section{As Leis de Kepler}
No início do século XVII, Johannes Kepler, utilizando os dados precisos de Tycho Brahe, formulou três leis empíricas que revolucionaram a astronomia:

\begin{enumerate}
    \item \textbf{Primeira Lei (Lei das Órbitas Elípticas):} Todos os planetas se movem em \textbf{órbitas elípticas}, com o Sol ocupando um dos focos da elipse. Isso significava um rompimento radical com a ideia secular de órbitas circulares.
    \item \textbf{Segunda Lei (Lei das Áreas Iguais):} Uma linha imaginária que liga o Sol a um planeta varre \textbf{áreas iguais em tempos iguais}. Essa lei implica que um planeta se move mais rapidamente quando está mais próximo do Sol (no periélio) e mais lentamente quando está mais distante (no afélio).
    \item \textbf{Terceira Lei (Lei Harmônica):} O quadrado do período orbital ($T$) de um planeta é \textbf{diretamente proporcional} ao cubo do semieixo maior ($a$) de sua órbita elíptica. Matematicamente, podemos expressar isso como:
    $$ T^2 \propto a^3 $$
    ou, em termos de uma constante de proporcionalidade ($K$):
    $$ T^2 = K a^3 $$
    Esta lei permitiu a comparação dos períodos e distâncias dos planetas no sistema solar.
\end{enumerate}

\section{A Gravitação Universal de Newton}
As leis de Kepler descreveram \textit{como} os planetas se moviam, mas foi Isaac Newton quem, com sua \textbf{lei da gravitação universal}, explicou \textit{por que} eles se moviam dessa forma. Newton postulou que toda partícula de matéria no universo atrai toda outra partícula com uma força que é diretamente proporcional ao produto de suas massas e inversamente proporcional ao quadrado da distância entre seus centros. A fórmula é:

$$ F = G \frac{m_1 m_2}{r^2} $$

Onde:
\begin{itemize}
    \item $F$ é a força gravitacional entre os dois objetos.
    \item $G$ é a \textbf{constante gravitacional universal} ($6.674 \times 10^{-11} \, \text{N} \cdot \text{m}^2/\text{kg}^2$).
    \item $m_1$ e $m_2$ são as massas dos dois objetos.
    \item $r$ é a distância entre os centros dos dois objetos.
\end{itemize}

A genialidade de Newton foi demonstrar que a mesma força que faz uma maçã cair da árvore é a responsável por manter a Lua em órbita ao redor da Terra e os planetas ao redor do Sol.

\section{A Síntese de Kepler e Newton}
A lei da gravitação universal de Newton forneceu a base física para as leis empíricas de Kepler. Ao aplicar o cálculo e sua lei da gravitação, Newton foi capaz de derivar as leis de Kepler, mostrando a profunda interconexão entre elas. A órbita elíptica dos planetas é uma consequência direta da força gravitacional que diminui com o quadrado da distância.

\section{Conclusão}
As órbitas planetárias são um testemunho da elegância das leis da física. Da engenhosidade observacional de Kepler à poderosa síntese teórica de Newton, nossa compreensão desses movimentos celestes não só nos permitiu prever posições planetárias, mas também abriu as portas para a exploração espacial e uma compreensão mais profunda do cosmos. A dança cósmica dos planetas é, em essência, uma manifestação visível da gravidade em ação.

\end{document}