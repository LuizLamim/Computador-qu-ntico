\documentclass{article}
\usepackage[utf8]{inputenc}
\usepackage{amsmath} % Necessário para ambientes matemáticos como 'align*'

\title{Relações Trigonométricas}
\author{} % Você pode adicionar seu nome aqui

\date{} % Para remover a data, ou use \today para a data atual

\begin{document}

\maketitle

\section*{Definições Básicas}
Considere um triângulo retângulo com um ângulo $\theta$. Os lados são definidos como:
\begin{itemize}
    \item \textbf{Oposto}: O lado em frente ao ângulo $\theta$.
    \item \textbf{Adjacente}: O lado próximo ao ângulo $\theta$ que não é a hipotenusa.
    \item \textbf{Hipotenusa}: O lado mais longo, oposto ao ângulo reto.
\end{itemize}

As relações trigonométricas fundamentais são:
\begin{align*}
    \sin(\theta) &= \frac{\text{Oposto}}{\text{Hipotenusa}} \\
    \cos(\theta) &= \frac{\text{Adjacente}}{\text{Hipotenusa}} \\
    \tan(\theta) &= \frac{\text{Oposto}}{\text{Adjacente}}
\end{align*}

---

\section*{Relações Recíprocas}
As relações recíprocas são os inversos das relações trigonométricas básicas:
\begin{align*}
    \csc(\theta) &= \frac{1}{\sin(\theta)} = \frac{\text{Hipotenusa}}{\text{Oposto}} \\
    \sec(\theta) &= \frac{1}{\cos(\theta)} = \frac{\text{Hipotenusa}}{\text{Adjacente}} \\
    \cot(\theta) &= \frac{1}{\tan(\theta)} = \frac{\text{Adjacente}}{\text{Oposto}}
\end{align*}

---

\section*{Identidades Pitagóricas}
As identidades pitagóricas são derivadas do Teorema de Pitágoras:
\begin{align*}
    \sin^2(\theta) + \cos^2(\theta) &= 1 \\
    1 + \tan^2(\theta) &= \sec^2(\theta) \\
    1 + \cot^2(\theta) &= \csc^2(\theta)
\end{align*}

---

\section*{Relações de Quociente}
\begin{align*}
    \tan(\theta) &= \frac{\sin(\theta)}{\cos(\theta)} \\
    \cot(\theta) &= \frac{\cos(\theta)}{\sin(\theta)}
\end{align*}

---

\section*{Sinais das Funções Trigonométricas nos Quadrantes}
\begin{itemize}
    \item \textbf{Quadrante I (0° a 90°)}: Todas as funções são positivas.
    \item \textbf{Quadrante II (90° a 180°)}: $\sin(\theta)$ e $\csc(\theta)$ são positivas.
    \item \textbf{Quadrante III (180° a 270°)}: $\tan(\theta)$ e $\cot(\theta)$ são positivas.
    \item \textbf{Quadrante IV (270° a 360°)}: $\cos(\theta)$ e $\sec(\theta)$ são positivas.
\end{itemize}

\end{document}