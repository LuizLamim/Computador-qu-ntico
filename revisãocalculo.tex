\documentclass[a4paper,11pt]{article}

% --- Pacotes Fundamentais ---
\usepackage[utf8]{inputenc}
\usepackage[T1]{fontenc}
\usepackage[brazil]{babel}
\usepackage{amsmath, amssymb, amsthm} % Símbolos matemáticos
\usepackage{geometry} % Margens
\usepackage{xcolor} % Cores para destaque
\usepackage{enumitem} % Personalização de listas
\usepackage{hyperref} % Links e metadados
\usepackage{titlesec} % Formatação de títulos

% --- Configuração de Página ---
\geometry{left=2.5cm, right=2.5cm, top=2.5cm, bottom=2.5cm}

% --- Cores Personalizadas ---
\definecolor{primaryBlue}{RGB}{0, 51, 102}
\definecolor{accentRed}{RGB}{153, 0, 0}

% --- Formatação dos Títulos ---
\titleformat{\section}
{\normalfont\Large\bfseries\color{primaryBlue}}{\thesection}{1em}{}
\titleformat{\subsection}
{\normalfont\large\bfseries\color{primaryBlue}}{\thesubsection}{1em}{}

% --- Metadados do PDF ---
\hypersetup{
    pdftitle={Plano de Revisão de Cálculo - Stewart},
    pdfauthor={Usuário},
    colorlinks=true,
    linkcolor=primaryBlue,
    urlcolor=primaryBlue
}

% --- Título e Cabeçalho ---
\title{\textbf{Plano de Revisão Estratégica: Cálculo 1, 2 e 3}\\
\large Guia de Manutenção e Aprofundamento (Baseado em James Stewart)}
\author{}
\date{\today}

\begin{document}

\maketitle
\thispagestyle{empty}

\noindent \textbf{Objetivo:} Como você já domina o conteúdo, este guia não é linear. Ele foca nos \textit{pilares conceituais} e na conexão lógica entre os tópicos, priorizando o rigor matemático e a visualização geométrica.

\hrulefill

% --------------------------------------------------------
\section{Fase 1: O Alicerce Rigoroso (Cálculo 1)}
\textit{Objetivo: Revisar a precisão das definições e teoremas fundamentais.}

\subsection{1. Limites e Continuidade (Cap. 2)}
\begin{itemize}[label=$\bullet$]
    \item \textbf{Foco:} Definição formal $\epsilon - \delta$ (apenas para treinar o rigor matemático) e Teorema do Valor Intermediário (TVI).
    \item \textbf{Ponto Crítico:} Revise os limites fundamentais trigonométricos e exponenciais, pois são cruciais para Séries de Taylor.
\end{itemize}

\subsection{2. A Derivada e Teoremas de Valor Médio (Cap. 3 e 4)}
\begin{itemize}[label=$\bullet$]
    \item \textbf{Foco:} Pule a mecânica básica. Vá direto para o \textbf{Teorema do Valor Médio (TVM)} e suas consequências.
    \item \textbf{Aplicações:} Taxas relacionadas e Otimização. Problemas que testam modelagem, não apenas cálculo.
    \item \textbf{Regra de L'Hôpital:} Essencial para análise de comportamento assintótico.
\end{itemize}
\begin{quote}
    \small \textbf{Desafio:} Tente provar o TVM mentalmente desenhando o gráfico. Se conseguir, dominou o conceito.
\end{quote}

% --------------------------------------------------------
