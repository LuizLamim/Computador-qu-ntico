\documentclass[a4paper,11pt]{article}

% --- Pacotes Fundamentais ---
\usepackage[utf8]{inputenc}
\usepackage[T1]{fontenc}
\usepackage[brazil]{babel}
\usepackage{amsmath, amssymb, amsthm} % Símbolos matemáticos
\usepackage{geometry} % Margens
\usepackage{xcolor} % Cores para destaque
\usepackage{enumitem} % Personalização de listas
\usepackage{hyperref} % Links e metadados
\usepackage{titlesec} % Formatação de títulos

% --- Configuração de Página ---
\geometry{left=2.5cm, right=2.5cm, top=2.5cm, bottom=2.5cm}

% --- Cores Personalizadas ---
\definecolor{primaryBlue}{RGB}{0, 51, 102}
\definecolor{accentRed}{RGB}{153, 0, 0}

% --- Formatação dos Títulos ---
\titleformat{\section}
{\normalfont\Large\bfseries\color{primaryBlue}}{\thesection}{1em}{}
\titleformat{\subsection}
{\normalfont\large\bfseries\color{primaryBlue}}{\thesubsection}{1em}{}

% --- Metadados do PDF ---
\hypersetup{
    pdftitle={Plano de Revisão de Cálculo - Stewart},
    pdfauthor={Usuário},
    colorlinks=true,
    linkcolor=primaryBlue,
    urlcolor=primaryBlue
}

% --- Título e Cabeçalho ---
\title{\textbf{Plano de Revisão Estratégica: Cálculo 1, 2 e 3}\\
\large Guia de Manutenção e Aprofundamento (Baseado em James Stewart)}
\author{}
\date{\today}

\begin{document}

\maketitle
\thispagestyle{empty}

\noindent \textbf{Objetivo:} Como você já domina o conteúdo, este guia não é linear. Ele foca nos \textit{pilares conceituais} e na conexão lógica entre os tópicos, priorizando o rigor matemático e a visualização geométrica.

\hrulefill

% --------------------------------------------------------
\section{Fase 1: O Alicerce Rigoroso (Cálculo 1)}
\textit{Objetivo: Revisar a precisão das definições e teoremas fundamentais.}

\subsection{1. Limites e Continuidade (Cap. 2)}
\begin{itemize}[label=$\bullet$]
    \item \textbf{Foco:} Definição formal $\epsilon - \delta$ (apenas para treinar o rigor matemático) e Teorema do Valor Intermediário (TVI).
    \item \textbf{Ponto Crítico:} Revise os limites fundamentais trigonométricos e exponenciais, pois são cruciais para Séries de Taylor.
\end{itemize}

\subsection{2. A Derivada e Teoremas de Valor Médio (Cap. 3 e 4)}
\begin{itemize}[label=$\bullet$]
    \item \textbf{Foco:} Pule a mecânica básica. Vá direto para o \textbf{Teorema do Valor Médio (TVM)} e suas consequências.
    \item \textbf{Aplicações:} Taxas relacionadas e Otimização. Problemas que testam modelagem, não apenas cálculo.
    \item \textbf{Regra de L'Hôpital:} Essencial para análise de comportamento assintótico.
\end{itemize}
\begin{quote}
    \small \textbf{Desafio:} Tente provar o TVM mentalmente desenhando o gráfico. Se conseguir, dominou o conceito.
\end{quote}

% --------------------------------------------------------
\section{Fase 2: Acumulação e Convergência (Cálculo 2)}
\textit{Objetivo: Dominar Séries Infinitas e a conexão Integral-Derivada.}

\subsection{3. Integração e o TFC (Cap. 5)}
\begin{itemize}[label=$\bullet$]
    \item \textbf{Foco:} A conexão geométrica da Soma de Riemann com a Integral Definida.
    \item \textbf{Conceito Chave:} O \textbf{Teorema Fundamental do Cálculo (TFC)}, partes 1 e 2. Entenda profundamente a relação inversa:
    \[ \frac{d}{dx} \int_{a}^{x} f(t) \, dt = f(x) \]
\end{itemize}

\subsection{4. Técnicas e Aplicações (Cap. 6, 7 e 8)}
\begin{itemize}[label=$\bullet$]
    \item \textbf{Técnicas:} Substituição Trigonométrica e Frações Parciais.
    \item \textbf{Aplicações:} Volume de sólidos (discos vs. cascas cilíndricas) e comprimento de arco (preparação para visualização 3D).
\end{itemize}

\subsection{5. Sequências e Séries (Cap. 11) -- \textcolor{accentRed}{Prioridade Alta}}
\begin{itemize}[label=$\bullet$]
    \item \textbf{Testes de Convergência:} Razão, Raiz e Comparação.
    \item \textbf{Séries de Potências:} Raio e intervalo de convergência.
    \item \textbf{Séries de Taylor e Maclaurin:} Reescreva $e^x$, $\sin(x)$ e $\cos(x)$ como séries:
    \[ f(x) = \sum_{n=0}^{\infty} \frac{f^{(n)}(a)}{n!} (x-a)^n \]
\end{itemize}

% --------------------------------------------------------
\section{Fase 3: O Espaço e a Generalização (Cálculo 3)}
\textit{Objetivo: Expandir para $\mathbb{R}^n$ e dominar o Cálculo Vetorial.}

\subsection{6. Geometria Analítica e Derivadas Parciais (Cap. 10, 12 e 14)}
\begin{itemize}[label=$\bullet$]
    \item \textbf{Vetores:} Produto escalar (trabalho/projeção) e vetorial (torque/normal).
    \item \textbf{O Vetor Gradiente ($\nabla f$):} Entenda-o geometricamente como a direção de maior crescimento e sua ortogonalidade às curvas de nível.
    \item \textbf{Otimização:} Multiplicadores de Lagrange (onde os gradientes são paralelos).
\end{itemize}

\subsection{7. Integrais Múltiplas (Cap. 15)}
\begin{itemize}[label=$\bullet$]
    \item \textbf{Foco:} Mudança de variáveis e o \textbf{Jacobiano}. Entenda o Jacobiano como fator de distorção de área/volume (ex: $r \, dr \, d\theta$).
\end{itemize}

\subsection{8. Cálculo Vetorial (Cap. 16) -- \textcolor{accentRed}{A Joia da Coroa}}
\begin{itemize}[label=$\bullet$]
    \item \textbf{Campos Vetoriais:} Conservativos vs. não conservativos ($\operatorname{rot} \mathbf{F}$ e $\operatorname{div} \mathbf{F}$).
    \item \textbf{Os Três Grandes Teoremas:}
    \begin{enumerate}
        \item \textbf{Teorema de Green:} Borda vs. Interior no plano.
        \item \textbf{Teorema de Stokes:} Circulação 3D.
        \[ \oint_C \mathbf{F} \cdot d\mathbf{r} = \iint_S \operatorname{rot} \mathbf{F} \cdot d\mathbf{S} \]
        \item \textbf{Teorema do Divergente (Gauss):} Fluxo através de superfície fechada.
    \end{enumerate}
\end{itemize}

\hrulefill

\section*{Como usar o Stewart nesta fase?}
Não faça os exercícios ímpares 1-50. Foque nas seções finais de cada capítulo:
\begin{itemize}
    \item \textbf{Verificação de Conceitos (Concept Check):} Responda oralmente para testar a teoria.
    \item \textbf{Problemas Quentes (Problems Plus):} Resolva 2 ou 3 desses por capítulo. Eles exigem criatividade e integram múltiplos conceitos.
\end{itemize}

\end{document}