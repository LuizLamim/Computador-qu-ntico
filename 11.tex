\documentclass{article}
\usepackage[utf8]{inputenc}
\usepackage{amsmath} % Required for mathematical equations

\title{Elipse: Uma Curva Cônica Fascinante}
\author{} % You can add your name here if you wish
\date{} % This will remove the date from the document

\begin{document}

\maketitle

\section*{Elipse: Uma Curva Cônica Fascinante}

A \textbf{elipse} é uma das mais belas e importantes curvas em matemática e física, pertencente à família das \textbf{seções cônicas}. Ela pode ser definida de várias maneiras, cada uma revelando aspectos distintos de suas propriedades e aplicações.

\subsection*{Definição Geométrica}

Geometricamente, uma elipse é o conjunto de todos os pontos em um plano para os quais a soma das distâncias a dois pontos fixos, chamados \textbf{focos} ($F_1$ e $F_2$), é constante. Essa constante é geralmente denotada por $2a$, onde $a$ é o comprimento do \textbf{semieixo maior}.

$$PF_1 + PF_2 = 2a$$

A distância entre os focos é $2c$. Os focos estão localizados no \textbf{eixo maior}, que é o segmento de reta que passa pelos focos e pelos vértices da elipse. O \textbf{eixo menor} é perpendicular ao eixo maior e passa pelo centro da elipse. O comprimento do semieixo menor é denotado por $b$.

\subsection*{Equação Canônica}

A equação canônica de uma elipse centrada na origem $(0,0)$ depende da orientação de seus eixos.

Se o eixo maior está ao longo do eixo $x$, a equação é:

$$\frac{x^2}{a^2} + \frac{y^2}{b^2} = 1$$

Se o eixo maior está ao longo do eixo $y$, a equação é:

$$\frac{x^2}{b^2} + \frac{y^2}{a^2} = 1$$

Onde $a > b > 0$. A relação entre $a$, $b$ e $c$ é dada por:

$$c^2 = a^2 - b^2$$

\subsection*{Excentricidade}

A \textbf{excentricidade} ($e$) de uma elipse é uma medida de quão "achatada" ela é, e é definida como a razão da distância focal para o comprimento do semieixo maior:

$$e = \frac{c}{a}$$

Para uma elipse, $0 < e < 1$. Quando $e$ se aproxima de 0, a elipse se aproxima de um círculo. Quando $e$ se aproxima de 1, a elipse se torna mais alongada.

\subsection*{Propriedades e Aplicações}

As elipses possuem diversas propriedades interessantes e são fundamentais em muitas áreas:

\begin{itemize}
    \item \textbf{Propriedade Refletora:} Um raio de luz ou som que emana de um foco de uma elipse, após ser refletido pela superfície da elipse, passará pelo outro foco. Essa propriedade é utilizada em designs de galerias de sussurros e em certos tipos de refletores.
    \item \textbf{Leis de Kepler:} As órbitas dos planetas ao redor do Sol são elipses, com o Sol em um dos focos. Essa foi uma descoberta revolucionária de Johannes Kepler, que mudou nossa compreensão do sistema solar.
    \item \textbf{Engenharia e Arquitetura:} Arcos elípticos são estruturalmente fortes e esteticamente agradáveis, sendo utilizados em pontes, cúpulas e abóbadas.
    \item \textbf{Óptica:} Lentes e espelhos elípticos são usados para focar luz em instrumentos ópticos.
\end{itemize}

A elipse, com sua elegância matemática e suas profundas implicações no mundo físico, continua a ser um tópico de estudo e admiração em diversas disciplinas.

\end{document}